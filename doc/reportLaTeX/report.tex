\documentclass[a4paper]{article}

\usepackage[english]{babel}
\usepackage[utf8]{inputenc}
\usepackage{amsmath}
\usepackage{graphicx}
\usepackage[colorinlistoftodos]{todonotes}
\usepackage{algorithm}
\usepackage{algorithmic}
\usepackage{listings}


\title{COMP90015 Distributed Systems \\Assignment 2: Distributed Shared White Board}

\author{ZHAOFENG QIU 1101584\\ ZHIHAN GUO 1100249\\ AOQI ZUO 1028089\\ SHIRU XIE 1077303}

\date{\today}

\begin{document}
\maketitle

\section{System Description}
\label{sec:description}

In this project, we design and implement a shared whiteboard system that can be edited simultaneously, using a P2P structure with a centralized index server. The system supports a range of features such as freehand drawing, drawing multiple shapes such as lines, circles, rectangles, and oval with specific colors and thicknesses. Undo and Redo method is also provided to the user. We also implement a "File" menu, which allows the manager to new, open, or save a file in different formats. Besides, the system also provides a chat window so that all the users in the same shared whiteboard can send messages to each other.  Chat content is encrypted when transmitted, which ensures users' privacy. Moreover, we provide a lobby system for our users to create, join, and search for specific whiteboard rooms, which we think is a very user-friendly improvement.

\section{Design Details}
\subsection{System Structure}
\label{sec:structure}
The whole system is implemented by a P2P with a Centralized Index Server Architecture. Although different users may have different permissions to operate on one artboard, they are essentially both servers and users, interacting cooperatively as peers to view and modify a canvas without distinction between clients and servers. There is also a central server saving connection information for each user and their whiteboard room. 

\subsection{Communication Mechanism}
\label{sec:communication}
The communication in this project takes place via TCP sockets and RMI. Specifically, the communications between the central server and all users are designed in TCP sockets with thread-per-request architecture. As for users, the connection and chat message exchange between host and guests are implemented by TCP sockets with thread-per-connection; the interactive modification to the whiteboard between users takes place via Two-way RMI. In the centralized server, a socket is listening for and accepting requests from clients. For each request, a thread is created to handle the request between a client and the server. 
\\
\\
TCP socket is greatly reliable, which guarantees the order of the arrivals of the request and the order of the response. 
\\
\\
Java RMI (Java Remote Method Invocation) is an application programming interface for implementing remote procedure calls in the Java programming language. It enables programs running on the client to call objects on the remote server. This mechanism brings great convenience to system design and programming of distributed computing.

\subsection{Concurrency}
\label{sec:concurrency}
In each room, the guest can have access to use the shared resources of the whiteboard. In other words, guests can draw simultaneously on a canvas. There is a range of features that the whiteboard has implemented, such as drawing line, circle, rectangle, and oval with various colors and various sizes.

\subsection{Message Exchange}
\label{sec:exchange}
The messages exchanged from users are encrypted to ensure confidentiality. We use the JSON format, which is more practical than passing messages by the string.

\subsection{Interface Utilities}
\label{sec:utilities}
In this system, we use the Java2D drawing package. Java 2D is an API for drawing two-dimensional graphics using the Java programming language. Java 2D is a powerful technology. It can be used to create rich user interfaces, games, animations, multimedia applications, or various special effects.

\section{Design Details}





\end{enumerate}
\end{document}













